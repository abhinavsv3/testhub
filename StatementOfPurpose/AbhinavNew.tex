
\documentclass[11pt]{article}

\usepackage{hyperref}
\hypersetup{
    colorlinks=true,
    linkcolor=black,
    filecolor=magenta,      
    urlcolor=blue,
}
\urlstyle{same}

%\documentstyle[11pt]{article}
\setlength{\oddsidemargin}{-.5in}
\setlength{\evensidemargin}{-.5in}
\setlength{\textwidth}{7.5in}
\setlength{\topmargin}{-3cm}
\setlength{\textheight}{9.5in}

\begin{document}
\begin{flushleft}{\textsc{Abhinav Shankaranarayanan Venkataraman} \hfill University of Florida -- Gainsville \newline Date of Birth : 12-Aug-1994  \hfill MS in Computer Science}\end{flushleft}
\begin{center}
{\LARGE Statement of Purpose}
\end{center}

I have been exposed to computer programming since my high school days.  I got a start at developing complex code at The Zonal Computing Olympiad of 2010 organized under the auspices of reputed \emph{Chennai Mathematical Institute (CMI)}. My effort brought appreciation from my teacher.   I am especially interested in algorithms, their application to real-life situations and the challenges they pose especially when the factor of Optimality is considered. The use of graph algorithms in network security, cryptography, distributed computing and other diverse areas of computer science have caught my attention.  I would like to explore these realms to a greater extent.
\\\par
At the \textbf{SASTRA University, Thanjavur} where I am pursuing a Bachelor's degree in Computer Science and Engineering (expected June 2016), my specific interests are in the areas of Algorithms and Artificial intelligence.  As preparation, I have learned mathematical concepts to handle combinatorial problems, randomized algorithms, range queries, game theory and distributed algorithms.
\\\par
I got an opportunity to explore these areas further as a research intern in summer 2014 at the \textbf{Indian Institute of Information Technology Design and Manufacturing (IIIT D\&M), Kancheepuram} as part of their \emph{``Research internship in Computer Science''} program in \emph{``Theoretical Computer Science''}.  During this program I worked under the guidance of Dr.N. Sadagopan, Assistant Professor of Computer Science. I was exposed to deterministic and non-deterministic algorithms, graph theory, algorithmic aspects of chordal graphs and NP Completeness. My specific research was a study on combinatorial problems on \emph{``Knight Graphs''} with particular focus on Hamiltonicity, Eulerian property and Colorability of the graph.
\\\par
Due to my commendable performance at IIIT D\&M, I was offered the \emph{visiting student} internship at the prestigious \textbf{The Institute of Mathematical Sciences (IMSc), Chennai} in their \textbf{Theoretical Computer Science} group during June 2015. I was among the top 20 students selected from the whole of India. Apart from attending the research lectures, I worked with a post-doctoral student of the institute in developing a 2-OPT solution for Harmonious coloring of a graph, which is an NP Complete problem. My final presentation at the institute was on \emph{``Chordal graphs and k-trees''}.
\\\par
With a strong grasp on core theoretical computer science, I also wanted to venture into its specific applications.I had the opportunity to learn about fuzzy logic and expert systems when I worked with Prof Krishna Anand S, Senior Associate Professor of Computer Science at SASTRA University, on a paper titled \textbf{``Effective management of bus transportation through design of a Fuzzy Expert System''} published in \emph{ARPN Journal of Engineering and Applied Sciences} URL:\url{http://www.arpnjournals.com/jeas/research_papers/rp_2015/jeas_0415_1773.pdf}
(accessible through Scopus). This paper proposes a cost effective high performance bus management system that minimizes the possibility of accidents, at the same time reducing wear and tear of tires. We built a fuzzy expert system for the purpose after a detailed analysis of the real time traffic patterns in the National Highway corridor of Thanjavur-Tiruchi (in the state of Tamil Nadu).
\\\par
In July of 2015, I was selected to represent my university as part of a Student Exchange program at \textbf{Tokyo City University (TCU), Japan} where I worked on algorithms that help to recognize edema of the brain given its X-Ray CT image. All of my highly enriching internship and research experiences have given me the impetus to strive towards a career where I can actively contribute to the field of computer science.
\\\par
My experiences and track records have also enabled me to secure a semester-long research internship during my final semester at the \textbf{Universitat Politecnica de Catalunya (UPC) BarcelonaTech, Barcelona, Spain} in the \emph{Laboratory for Relational Algorithmics, Complexity and Learning (LARCA)} under \emph{Prof.Ricard Gavalda} and \emph{Prof.Marta Arias}. This opportunity will allow me to complete my undergraduate final year project-thesis through the Semester Abroad Program (SAP) offered by my university. 
\\\par
I have been a diligent student throughout my academic career at SASTRA University, always being among the top ten percent of my batch (Dean's List). I have also secured first rank in ``Linux programming'' course.  My interest to explore topics beyond the basic curriculum has also motivated me to actively contribute to the open source organization GNOME. I have written two approved minor patches for their calculator project in 2014. My passion for programming drives me to consistently participate in coding competitions like the ACM ICPC- the International Collegiate Programming Contest organized by the ACM. 
\\\par
My sense of responsibility by way of passing on acquired knowledge has further encouraged me to organize coding, web development and networking clubs in my capacity as the General Secretary of ACE (Association of Computer Engineers), at SASTRA University. I also give frequent lectures on Algorithms, Data structures, C and C++ to Juniors and Sophomores.
\\\par
As someone incharge of the computing cluster of Daksh 2016 -the second biggest Techno management fest of South India, hosted by SASTRA University, I have the opportunity to hone my managerial and interpersonal skills. My responsibilities include working with dynamic teams to design, plan and implement high standard competitions and events to cater to a participating audience of more than 1000 students nationwide. 
\\\par
I strongly believe in giving back to the society. I have been involved in community service activities such as aiding rural development.  I have been of help to the Center for Rural Development Studies(CRDS),SASTRA University, Thanjavur. My most significant contribution has been the development of a mobile application called ``Jagran'' at the Accenture mobility lab at SASTRA, which is an information providing app for villagers. I have also designed a weed-removing robot as part of the e-yantra contest organized by the \emph{Indian Institute of Technology (IIT), Bombay}.
\\\par
My future goal is to be actively involved in research and solve problems that bring significant impact to society. I believe I have an opportunity to prepare adequately for the future by enrolling in Graduate studies in Computer Science at the University of Florida -- Gainsville especially in the field of Theoretical Computer Science, Algorithms and Artificial Intelligence. I am particularly interested in the work of \emph{Prof.Sartaj K Sahni} in \emph{``Algorithms and data structures and medical Algorithms''}, \emph{Prof.Dr.Meera Sitharam} in \emph{``Algorithms and Machine learning''} and I am interested in joinging the "Geoplexity Group". I believe that with my diverse experiences, coupled with my enthusiasm to learn more, I can be a part of interesting innovations in the field of computer science.
\\\par
In summary, I am an enterprising and hard-working student with an open minded and curiosity driven attitude.  I would like to be provided a strong platform to steadily progress towards my stated aspirations especially in the domain of Computer Science.  I believe that my track record thus far and my self-confidence do provide the needed foundation.  I sincerely hope to be part of the learning ecosystem at the University of Florida -- Gainsville.


\end{document}
